\documentclass[a4paper]{article}
\usepackage[UTF8]{ctex}
\usepackage{amssymb}
\usepackage{amsmath}
\usepackage[margin=2.5cm]{geometry}  % 缩小边距

\begin{document}

\section*{IMO 2009; B1 EX15}
    \textbf{Problem:}
    Let $n$ be a positive integer and let $a_1,\ldots,a_k$ $(k \geq 2)$ be distinct integers in the set $\{1,\ldots,n\}$ such that 
    \[n \mid a_i(a_{i+1}-1) \quad \text{for } i = 1,\ldots,k-1.\]
    Prove that 
    \[n \nmid a_k(a_1-1).\]

    \[ \]
    设 $n$ 是一个正整数,$n > 1$,$a_0, \ldots, a_k$ $(k \geq 1)$ 是集合 $\{1, \ldots, n\}$ 中的互不相同的整数,满足
    $n \mid a_i(a_{i+1}-1)$,其中 $i = 0, \ldots, k-1$。证明:
    $n \nmid a_k(a_0-1)$。

\section*{Lean New Proof}
    \textbf{Lean New Proof:}

    避免 $i,i+1$ 范围混乱
    令 $t = i + 1,其中i = 0, \ldots, k-1; t= 1, \ldots, $,

    1. 根据条件 $n \mid a_i(a_{i+1}-1)$,$i = 0,\ldots,k-1$ 我们有:
    $n \mid a_0(a_1 - 1)$ 因此 令 $(n,a_0) = p,q =\frac{n}{p}$亦为整数,则有 $n = pq$,
    并且 $p \mid a_0,q \mid a_1 - 1$.故 $(q,a_1) = 1$.
    
    2. 由 $q \mid a_1 - 1$,可得 $q \mid a_2 - 1$,通过归纳法,我们可以证明对于所有 $i = 1, \ldots, k$,都有 $q \mid a_i - 1$。
    因此 对于任意的 $i = 1, \ldots, k$,都有 $q \mid a_i - 1$。
    特别的 ,因为 $q \mid a_0$,所以 $n = pq \mid a_0(a_k - 1)$。
    
    3. 反之,若结论不成立,则 $n \mid a_k(a_0 - 1)$, 与$n = pq \mid a_0(a_k - 1)$相减可得 $n \mid (a_k - a_0)$.矛盾
    总上述, $n \nmid a_k(a_0 - 1)$.
    



    1. 根据条件 $n \mid a_i(a_{i+1}-1)$,$i = 0,\ldots,k-1$ 我们有:
    $n \mid a_0(a_1 - 1)$ 因此 令 $(n,a_0) = p,q =\frac{n}{p}$亦为整数,则有 $n = pq$,
    并且 $p \mid a_0,q \mid a_1 - 1$.故 $(q,a_1) = 1$.
    
    2. 由 $q \mid a_1 - 1$,可得 $q \mid a_2 - 1$,通过归纳法,我们可以证明对于所有 $i = 1, \ldots, k$,都有 $q \mid a_i - 1$。
    因此 对于任意的 $i = 1, \ldots, k$,都有 $q \mid a_i - 1$。
    特别的 ,因为 $q \mid a_0$,所以 $n = pq \mid a_0(a_k - 1)$。
    
    3. 反之,若结论不成立,则 $n \mid a_k(a_0 - 1)$, 与$n = pq \mid a_0(a_k - 1)$相减可得 $n \mid (a_k - a_0)$.矛盾
    总上述, $n \nmid a_k(a_0 - 1)$.



\section*{Lean New Proof}
    \textbf{Lean New Proof:}
    
    1. 根据条件 $n \mid a_i(a_{i+1}-1)$,我们有:
    $n \mid a_1(a_2 - 1)$ 因此 令 $(n,a_1) = p,q =\frac{n}{p}$亦为整数,则有 $n = pq$,
    并且 $p \mid a_1,q \mid a_2 - 1$.故 $(q,a_2) = 1$.
    
    2. 由 $q \mid a_2 - 1$,可得 $q \mid a_3 - 1$,通过归纳法,我们可以证明对于所有 $i = 2, \ldots, k$,都有 $q \mid a_i - 1$。
    因此 对于任意的 $i = 2, \ldots, k$,都有 $q \mid a_i - 1$。
    特别的 ,因为 $q \mid a_1$,所以 $n = pq \mid a_1(a_k - 1)$。
    
    3. 反之,若结论不成立,则 $n \mid a_k(a_1 - 1)$, 与$n = pq \mid a_1(a_k - 1)$相减可得 $n \mid (a_k - a_1)$.矛盾
    总上述, $n \nmid a_k(a_1 - 1)$.

\section*{Proof}
    \textbf{Proof:}

    反证法: 假设 $n \mid a_k(a_1-1)$。令 $n = pq$,其中 $p$ 和 $q$ 是互质正整数。

    \textbf{步骤 1:分析整除条件。}  
    根据条件 $n \mid a_i(a_{i+1}-1)$,我们有:
    $p \mid a_i$ 且 $q \mid a_{i+1}-1$,其中 $i = 1, \ldots, k-1$。
    由于 $q \mid a_2-1$,可得 $\gcd(q, a_2) = 1$,因此 $q \mid a_3-1$。通过归纳法,我们有:
    $q \mid a_i-1$,对于所有 $i = 2, \ldots, k$。

    \textbf{步骤 2:对 $a_1$ 的推论。}  
    由 $p \mid a_1$ 和 $q \mid a_1-1$,可得 $\gcd(p, q) = 1$。因此:
    $p \mid a_k$ 且 $q \mid a_k-1$。

    \textbf{步骤 3:矛盾。}  
    整数 $a_1, a_2, \ldots, a_k$ 是互不相同的,并且位于集合 $\{1, 2, \ldots, n\}$ 中。
    然而,条件 $p \mid a_i$ 和 $q \mid a_i-1$ 表明,在该范围内至多只有一个整数 $a_i$ 同时满足这两个条件。
    这与假设 $k \geq 2$ 矛盾。

    因此,我们的假设 $n \mid a_k(a_1-1)$ 是错误的。由此可得:
    $n \nmid a_k(a_1-1)$。

\section*{Lean Proof lemma}
    \textbf{Lean Proof:}

    下面是使用 Lean 证明的代码片段:

    1. 根据条件 $n \mid a_i(a_{i+1}-1)$,我们有:
    $p \mid a_i$ 且 $q \mid a_{i+1}-1$,其中 $i = 1, \ldots, k-1$。

    2. 通过归纳法,我们可以证明对于所有 $i = 2, \ldots, k$,都有 $q \mid a_i-1$。

    3. 由 $p \mid a_1$ 和 $q \mid a_1-1$,可得 $\gcd(p, q) = 1$

    4. $p \mid a_k$ 且 $q \mid a_k-1$

    5. 条件 $p \mid a_i$ 和 $q \mid a_i-1$ 表明,在该范围内至多只有一个整数 $a_i$ 同时满足这两个条件.
    即 $p \mid a_1, q \mid a_1-1$ 或 $p \mid a_k, q \mid a_k-1$



    \end{document}