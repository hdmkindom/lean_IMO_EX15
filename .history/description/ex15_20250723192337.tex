\documentclass[a4paper]{article}
\usepackage[UTF8]{ctex}
\usepackage{amssymb}
\usepackage{amsmath}
\usepackage[margin=2.5cm]{geometry}  % 缩小边距

\begin{document}

\section*{IMO 2009; B1 EX15}
    \textbf{Problem:}
    Let $n$ be a positive integer and let $a_1,\ldots,a_k$ $(k \geq 2)$ be distinct integers in the set $\{1,\ldots,n\}$ such that 
    \[n \mid a_i(a_{i+1}-1) \quad \text{for } i = 1,\ldots,k-1.\]
    Prove that 
    \[n \nmid a_k(a_1-1).\]

    \[ \]
    设 $n$ 是一个正整数,$a_1, \ldots, a_k$ $(k \geq 2)$ 是集合 $\{1, \ldots, n\}$ 中的互不相同的整数,满足
    $n \mid a_i(a_{i+1}-1)$,其中 $i = 1, \ldots, k-1$。证明:
    $n \nmid a_k(a_1-1)$。


\section*{Proof}

    \textbf{原证明:}

    Let $n=pq$ such that $p\mid a_1$ and $q\mid a_2-1$.
    Suppose $n$ divides $a_k(a_1-1)$. Note $q\mid a_2-1$ implies $(q,a_2)=1$ and hence $q\mid a_3-1$. 
    Similarly one has $q\mid a_i-1$ for all $i$ 's, in particular, 
    $p\mid a_1$ and $q\mid a_1-1$ force $(p,q)=1$. 
    Now $(p,a_1-1)=1$ gives $p\mid a_k$, similarly one has $p\mid a_i$ for all $i$ 's, that is $a_i$ 's satisfy $p\mid a_i$ and $q\mid a_i-1$, 
    but there should be at most one such integer satisfies them within the range of $1,2,\ldots,n$ for $n=pq$ and $(p,q)=1$. 
    A contradiction!!!

    \[\]
    \textbf{分步原证明:}

    We proceed by contradiction. Assume that $n \mid a_k(a_1-1)$. Let $n = pq$, where $p$ and $q$ are coprime positive integers.

    \textbf{Step 1: Analyze the divisibility condition.}  
    From the condition $n \mid a_i(a_{i+1}-1)$, we know:
    \[p \mid a_i \quad \text{and} \quad q \mid a_{i+1}-1 \quad \text{for } i = 1,\ldots,k-1.\]
    Since $q \mid a_2-1$, it follows that $\gcd(q, a_2) = 1$. Hence, $q \mid a_3-1$. By induction, we have:
    \[q \mid a_i-1 \quad \text{for all } i = 2,\ldots,k.\]

    \textbf{Step 2: Implications for $a_1$.}  
    From $p \mid a_1$ and $q \mid a_1-1$, we deduce that $\gcd(p, q) = 1$. Thus:
    \[p \mid a_k \quad \text{and} \quad q \mid a_k-1.\]

    \textbf{Step 3: Contradiction.}  
    The integers $a_1, a_2, \ldots, a_k$ are distinct and lie in the range $\{1, 2, \ldots, n\}$. 
    However, the conditions $p \mid a_i$ and $q \mid a_i-1$ imply that there can be at most one such integer $a_i$ satisfying both conditions within this range. 
    This contradicts the assumption that $k \geq 2$. 

    Therefore, our assumption that $n \mid a_k(a_1-1)$ must be false. Hence:
    \[n \nmid a_k(a_1-1).\]

    \[\]
    \textbf{中文翻译:}

    反证法: 假设 $n \mid a_k(a_1-1)$。令 $n = pq$,其中 $p$ 和 $q$ 是互质的正整数。

    \textbf{步骤 1:分析整除条件。}  
    根据条件 $n \mid a_i(a_{i+1}-1)$,我们有:
    $p \mid a_i$ 且 $q \mid a_{i+1}-1$,其中 $i = 1, \ldots, k-1$。
    由于 $q \mid a_2-1$,可得 $\gcd(q, a_2) = 1$,因此 $q \mid a_3-1$。通过归纳法,我们有:
    $q \mid a_i-1$,对于所有 $i = 2, \ldots, k$。

    \textbf{步骤 2:对 $a_1$ 的推论。}  
    由 $p \mid a_1$ 和 $q \mid a_1-1$,可得 $\gcd(p, q) = 1$。因此:
    $p \mid a_k$ 且 $q \mid a_k-1$。

    \textbf{步骤 3:矛盾。}  
    整数 $a_1, a_2, \ldots, a_k$ 是互不相同的,并且位于集合 $\{1, 2, \ldots, n\}$ 中。
    然而,条件 $p \mid a_i$ 和 $q \mid a_i-1$ 表明,在该范围内至多只有一个整数 $a_i$ 同时满足这两个条件。
    这与假设 $k \geq 2$ 矛盾。

    因此,我们的假设 $n \mid a_k(a_1-1)$ 是错误的。由此可得:
    $n \nmid a_k(a_1-1)$。

\end{document}